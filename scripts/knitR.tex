% Options for packages loaded elsewhere
\PassOptionsToPackage{unicode}{hyperref}
\PassOptionsToPackage{hyphens}{url}
%
\documentclass[
]{article}
\usepackage{lmodern}
\usepackage{amssymb,amsmath}
\usepackage{ifxetex,ifluatex}
\ifnum 0\ifxetex 1\fi\ifluatex 1\fi=0 % if pdftex
  \usepackage[T1]{fontenc}
  \usepackage[utf8]{inputenc}
  \usepackage{textcomp} % provide euro and other symbols
\else % if luatex or xetex
  \usepackage{unicode-math}
  \defaultfontfeatures{Scale=MatchLowercase}
  \defaultfontfeatures[\rmfamily]{Ligatures=TeX,Scale=1}
\fi
% Use upquote if available, for straight quotes in verbatim environments
\IfFileExists{upquote.sty}{\usepackage{upquote}}{}
\IfFileExists{microtype.sty}{% use microtype if available
  \usepackage[]{microtype}
  \UseMicrotypeSet[protrusion]{basicmath} % disable protrusion for tt fonts
}{}
\makeatletter
\@ifundefined{KOMAClassName}{% if non-KOMA class
  \IfFileExists{parskip.sty}{%
    \usepackage{parskip}
  }{% else
    \setlength{\parindent}{0pt}
    \setlength{\parskip}{6pt plus 2pt minus 1pt}}
}{% if KOMA class
  \KOMAoptions{parskip=half}}
\makeatother
\usepackage{xcolor}
\IfFileExists{xurl.sty}{\usepackage{xurl}}{} % add URL line breaks if available
\IfFileExists{bookmark.sty}{\usepackage{bookmark}}{\usepackage{hyperref}}
\hypersetup{
  pdftitle={knitR},
  pdfauthor={Bart Aben},
  hidelinks,
  pdfcreator={LaTeX via pandoc}}
\urlstyle{same} % disable monospaced font for URLs
\usepackage[margin=1in]{geometry}
\usepackage{color}
\usepackage{fancyvrb}
\newcommand{\VerbBar}{|}
\newcommand{\VERB}{\Verb[commandchars=\\\{\}]}
\DefineVerbatimEnvironment{Highlighting}{Verbatim}{commandchars=\\\{\}}
% Add ',fontsize=\small' for more characters per line
\usepackage{framed}
\definecolor{shadecolor}{RGB}{248,248,248}
\newenvironment{Shaded}{\begin{snugshade}}{\end{snugshade}}
\newcommand{\AlertTok}[1]{\textcolor[rgb]{0.94,0.16,0.16}{#1}}
\newcommand{\AnnotationTok}[1]{\textcolor[rgb]{0.56,0.35,0.01}{\textbf{\textit{#1}}}}
\newcommand{\AttributeTok}[1]{\textcolor[rgb]{0.77,0.63,0.00}{#1}}
\newcommand{\BaseNTok}[1]{\textcolor[rgb]{0.00,0.00,0.81}{#1}}
\newcommand{\BuiltInTok}[1]{#1}
\newcommand{\CharTok}[1]{\textcolor[rgb]{0.31,0.60,0.02}{#1}}
\newcommand{\CommentTok}[1]{\textcolor[rgb]{0.56,0.35,0.01}{\textit{#1}}}
\newcommand{\CommentVarTok}[1]{\textcolor[rgb]{0.56,0.35,0.01}{\textbf{\textit{#1}}}}
\newcommand{\ConstantTok}[1]{\textcolor[rgb]{0.00,0.00,0.00}{#1}}
\newcommand{\ControlFlowTok}[1]{\textcolor[rgb]{0.13,0.29,0.53}{\textbf{#1}}}
\newcommand{\DataTypeTok}[1]{\textcolor[rgb]{0.13,0.29,0.53}{#1}}
\newcommand{\DecValTok}[1]{\textcolor[rgb]{0.00,0.00,0.81}{#1}}
\newcommand{\DocumentationTok}[1]{\textcolor[rgb]{0.56,0.35,0.01}{\textbf{\textit{#1}}}}
\newcommand{\ErrorTok}[1]{\textcolor[rgb]{0.64,0.00,0.00}{\textbf{#1}}}
\newcommand{\ExtensionTok}[1]{#1}
\newcommand{\FloatTok}[1]{\textcolor[rgb]{0.00,0.00,0.81}{#1}}
\newcommand{\FunctionTok}[1]{\textcolor[rgb]{0.00,0.00,0.00}{#1}}
\newcommand{\ImportTok}[1]{#1}
\newcommand{\InformationTok}[1]{\textcolor[rgb]{0.56,0.35,0.01}{\textbf{\textit{#1}}}}
\newcommand{\KeywordTok}[1]{\textcolor[rgb]{0.13,0.29,0.53}{\textbf{#1}}}
\newcommand{\NormalTok}[1]{#1}
\newcommand{\OperatorTok}[1]{\textcolor[rgb]{0.81,0.36,0.00}{\textbf{#1}}}
\newcommand{\OtherTok}[1]{\textcolor[rgb]{0.56,0.35,0.01}{#1}}
\newcommand{\PreprocessorTok}[1]{\textcolor[rgb]{0.56,0.35,0.01}{\textit{#1}}}
\newcommand{\RegionMarkerTok}[1]{#1}
\newcommand{\SpecialCharTok}[1]{\textcolor[rgb]{0.00,0.00,0.00}{#1}}
\newcommand{\SpecialStringTok}[1]{\textcolor[rgb]{0.31,0.60,0.02}{#1}}
\newcommand{\StringTok}[1]{\textcolor[rgb]{0.31,0.60,0.02}{#1}}
\newcommand{\VariableTok}[1]{\textcolor[rgb]{0.00,0.00,0.00}{#1}}
\newcommand{\VerbatimStringTok}[1]{\textcolor[rgb]{0.31,0.60,0.02}{#1}}
\newcommand{\WarningTok}[1]{\textcolor[rgb]{0.56,0.35,0.01}{\textbf{\textit{#1}}}}
\usepackage{graphicx,grffile}
\makeatletter
\def\maxwidth{\ifdim\Gin@nat@width>\linewidth\linewidth\else\Gin@nat@width\fi}
\def\maxheight{\ifdim\Gin@nat@height>\textheight\textheight\else\Gin@nat@height\fi}
\makeatother
% Scale images if necessary, so that they will not overflow the page
% margins by default, and it is still possible to overwrite the defaults
% using explicit options in \includegraphics[width, height, ...]{}
\setkeys{Gin}{width=\maxwidth,height=\maxheight,keepaspectratio}
% Set default figure placement to htbp
\makeatletter
\def\fps@figure{htbp}
\makeatother
\setlength{\emergencystretch}{3em} % prevent overfull lines
\providecommand{\tightlist}{%
  \setlength{\itemsep}{0pt}\setlength{\parskip}{0pt}}
\setcounter{secnumdepth}{-\maxdimen} % remove section numbering

\title{knitR}
\author{Bart Aben}
\date{7/28/2020}

\begin{document}
\maketitle

\hypertarget{bulleted-list}{%
\subsubsection{Bulleted list:}\label{bulleted-list}}

\begin{itemize}
\tightlist
\item
  bullet 1
\item
  bullet 2
\item
  bullet 3

  \begin{itemize}
  \tightlist
  \item
    bullet 3a
  \item
    bullet 3b
  \end{itemize}
\end{itemize}

\hypertarget{numbered-list}{%
\subsubsection{Numbered list:}\label{numbered-list}}

\begin{enumerate}
\def\labelenumi{\arabic{enumi}.}
\tightlist
\item
  step 1
\item
  step 2
\item
  step 3
\end{enumerate}

\textbf{bold}

\emph{italics}

\hypertarget{title}{%
\section{Title}\label{title}}

\hypertarget{main-section}{%
\subsection{Main section}\label{main-section}}

\hypertarget{sub-section}{%
\subsubsection{Sub-section}\label{sub-section}}

\hypertarget{sub-sub-section}{%
\paragraph{Sub-sub section}\label{sub-sub-section}}

\hypertarget{hyperlink}{%
\paragraph{Hyperlink}\label{hyperlink}}

\href{http://www.tue.nl}{hyperlink}

\hypertarget{image}{%
\paragraph{Image}\label{image}}

\begin{figure}
\centering
\includegraphics{https://www.win.tue.nl/~opapapetrou/img/tue.png}
\caption{iamgeCaption}
\end{figure}

\hypertarget{subscript}{%
\paragraph{Subscript}\label{subscript}}

F\textsubscript{2}

\hypertarget{superscript}{%
\paragraph{Superscript}\label{superscript}}

F\textsuperscript{2}

\hypertarget{math-equations}{%
\paragraph{Math equations}\label{math-equations}}

\(E = mc^2\) \[y = \mu + \sum_{i=1}^p \beta_i x_i + \epsilon\]

\hypertarget{block-quote}{%
\paragraph{Block quote}\label{block-quote}}

A friend once said:

\begin{quote}
It's always better to give than to receive.
\end{quote}

\hypertarget{r-code-block}{%
\paragraph{R code block}\label{r-code-block}}

\begin{Shaded}
\begin{Highlighting}[]
\KeywordTok{summary}\NormalTok{(cars}\OperatorTok{$}\NormalTok{dist) }\CommentTok{# What if I add this really looooooooooooooooooooooooooooooooooooooong line????????????????????????????}
\end{Highlighting}
\end{Shaded}

\begin{verbatim}
##    Min. 1st Qu.  Median    Mean 3rd Qu.    Max. 
##    2.00   26.00   36.00   42.98   56.00  120.00
\end{verbatim}

\begin{Shaded}
\begin{Highlighting}[]
\KeywordTok{summary}\NormalTok{(cars}\OperatorTok{$}\NormalTok{speed)}
\end{Highlighting}
\end{Shaded}

\begin{verbatim}
##    Min. 1st Qu.  Median    Mean 3rd Qu.    Max. 
##     4.0    12.0    15.0    15.4    19.0    25.0
\end{verbatim}

\hypertarget{inline-r-code}{%
\paragraph{Inline R Code}\label{inline-r-code}}

There were 50 cars studied

\hypertarget{chunk-options}{%
\paragraph{Chunk options}\label{chunk-options}}

\begin{itemize}
\tightlist
\item
  Use \texttt{echo=FALSE} to avoid having the code itself shown.
\item
  Use \texttt{results="hide"} to avoid having any results printed.
\item
  Use \texttt{eval=FALSE} to have the code shown but not evaluated.
\item
  Use \texttt{warning=FALSE} and \texttt{message=FALSE} to hide any
  warnings or messages produced.
\item
  Use \texttt{fig.height} and \texttt{fig.width} to control the size of
  the figures produced (in inches).
\end{itemize}

\hypertarget{global-chunk-options}{%
\paragraph{Global chunk options}\label{global-chunk-options}}

\end{document}
